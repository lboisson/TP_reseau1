\documentclass[10pt,a4paper,final]{article}
\usepackage[utf8]{inputenc}
\usepackage[french]{babel}
\usepackage[T1]{fontenc}
\usepackage{amsmath}
\usepackage{amsfonts}
\usepackage{amssymb}
\author{Léo Boisson}
\setlength\parindent{24pt}
\title{\textbf{Compte rendu TP2/3 Réseau}}
\makeindex

\begin{document}
\date{}
\maketitle  
\tableofcontents  
\newpage

\section{Introduction} 

	\subsection{Historique}
		Les sockets sont un ensemble de fonctions de communications, proposés en 1980 pour le \textbf{Berkeley Software Distribution} (BSD), en open source, par 								l'université de Berkeley. Elles permettent a des applications de se connecter entre elles, via un principe client/serveur.	\\
		Aujourd'hui, les sockets sont disponibles dans quasiment tous les langages de programmations, et font offices de norme.\\
		On distingue deux modes de communication avec les sockets :\\
		\begin{itemize}
		\item
			Le mode connecté, qui utilise le protocole TCP. Dans ce mode, une connexion durable est établie entre les deux processus, afin que l'adresse de destination ne 					soit pas nécessaire à chaque envoie de données.
		\item
			Le mode non-connecté, qui utilise le protocole UDP. Ce mode nécessite l'adresse de destination à chaque envoi, et il n'y a pas de confirmation du bon envoi des
			données. Ce mode est plus adapté à l'envoi de flux audio ou vidéo.
		\end{itemize}+
		
	\subsection{Objectif du TP}
		L'objectif de ce TP est d'aborder le développement de sockets, et de se familiariser avec les outils qui vont avec (les primitives). Pour cela, nous allons 
		programmer des applications client/serveur basique, pour ensuite observer et analyser les échanges de données entre ces applications.
	
	
\section{Application client/serveur echo}


\section{Application client/serveur chifoumi}


\section{Échange de données entre deux machines}


\section{Analyse par Wireshark}


\section{conclusion}


\end{document}s